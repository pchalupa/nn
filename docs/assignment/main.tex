\documentclass{article}
\usepackage{enumitem}
\usepackage{csquotes}
\usepackage[english]{babel}
\usepackage[style=iso-authoryear,backend=biber]{biblatex}
\usepackage{setspace}

% Formatting Settings
\onehalfspacing
\addbibresource{bibliography.bib}

\title{Local-First Architecture in the Context of Web Applications}
\begin{document}
	\begin{titlepage}
		\begin{center}
			\textbf{UNICORN VYSOKÁ ŠKOLA S.R.O.}
			\vfill
			\textbf{\LARGE Local-First Architecture in the Context of Web Applications\\}
			\vspace{5mm}
			\textbf{Local-First architektura v kontextu webových aplikací}
			\vfill
			\textbf{Author: Petr Chalupa}
			\vfill
			\textbf{Supervisor: Ing. Michal Kökörčený, Ph.D.}
			\vfill
			\textbf{Prague, 2024}
			\vfill
		\end{center}
	\end{titlepage}
	\newpage
	\section{The Aim of this Thesis}
	The objective of this thesis is to provide a theoretical overview of local-first architecture in the context of web applications. It will describe the characteristics of local-first architecture and analyze its advantages and disadvantages compared to the traditional three-layered architecture. The theoretical part will also cover key topics such as Conflict-Free Replicated Data Types (CRDTs), replication strategies, and consistency models.

	The practical part of the thesis will focus on designing and implementing a framework that adheres to local-first principles. This framework will be demonstrated through a sample web application: a simple task manager in the form of a kanban board. The application will support offline functionality, synchronize data seamlessly when a connection is restored, and enable real-time collaboration.

	\section{Structure}
	\begin{enumerate}
		\item Introduction
		\item Theoretical Part
		\begin{enumerate}[label=\arabic{enumi}.\arabic*]
			\item Key Principles of Local-First Software
			\item Comparison with Three-Layered Architecture
			\item Conflict-Free Replicated Data Types
			\item Replication Strategies
			\item Consistency Models
		\end{enumerate}
		\item Practical Part
		\begin{enumerate}[label=\arabic{enumi}.\arabic*]
			\item Description of the Example Project
			\item Framework Architecture
			\item Testing Strategy
		\end{enumerate}
		\item Conclusion
	\end{enumerate}
	\nocite{*}
	\printbibliography[title={Literature},heading=bibnumbered]
	\section{Time Plan}
	\begin{enumerate}
		\item December 2024: Submit the Assignment
		\item January 2026: Submit the Thesis
	\end{enumerate}
\end{document}
