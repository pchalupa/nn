\documentclass{article}
\usepackage{enumitem}
\usepackage{todonotes}
\usepackage{csquotes}
\usepackage[english]{babel}
\usepackage[style=iso-authoryear,backend=biber]{biblatex}

\addbibresource{bibliography.bib}

\title{Local-First Architecture at the Context of Web Applications}
\begin{document}
	\begin{titlepage}
		\begin{center}
			\textbf{UNICORN VYSOKÁ ŠKOLA S.R.O.}
			\vfill
			\textbf{\LARGE Local-First Architecture in the Context of Web Applications\\}
			\vspace{5mm}
			\textbf{Local-First Architektura v kontextu webových aplikací}
			\vfill
			\textbf{Author: Petr Chalupa}
			\vfill
			\textbf{Supervisor: \todo{Add supervisor}}
			\vfill
			\textbf{Prague, 2024}
			\vfill
		\end{center}
	\end{titlepage}
	\section{Aim of this Thesis}
	The aim of this thesis is to provide a theoretical overview of local-first architecture in the context of web applications. It will describe the characteristics and requirements of this architecture, as well as its advantages and disadvantages when compared to the traditional three-layered architecture.\\
	The practical part of the thesis will focus on building a framework that adheres to local-first principles, which will be demonstrated through a sample web application.
	\section{Structure}
	\begin{enumerate}
		\item Introduction
		\item Theoretical part
		\begin{enumerate}[label=\arabic{enumi}.\arabic*]
			\item Characteristics of Local-First Architecture
			\item Comparison with Three-Layered Architecture
			\item Typical Use Cases
		\end{enumerate}
		\item Practical part
		\begin{enumerate}[label=\arabic{enumi}.\arabic*]
			\item Framework Design
			\item Example Web Application
		\end{enumerate}
		\item Conclusion
	\end{enumerate}
	\nocite{*}
	\printbibliography[title={Literature},heading=bibnumbered]
	\section{Time Plan}
	\begin{enumerate}
		\item November 2024: Submit the Assignment
		\item June 2025: Submit the Thesis
	\end{enumerate}
\end{document}
